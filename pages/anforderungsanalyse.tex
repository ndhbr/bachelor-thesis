\section{Anforderungsanalyse}\label{anforderungsanalyse}
Funktionale Anforderungen beschreiben, was das Projekt tun soll. Diese
Informationen sind wichtig, um die richtigen Werkzeuge und Programme für die
Umsetzung auszuwählen. Nichtfunktionale Bedingungen sind Bedingungen, wie zum
Beispiel Zuverlässigkeit, oder diverse Sicherheitsanforderungen.
\subsection{Funktionale Anforderungen}\label{anforderungsanalyse-funktional}
\subsubsection{Studierende}\label{anforderungsanalyse-funktional-stud}
Studierende sollen eine Kursübersicht mit allen Aufgaben haben. Der Fortschritt
von jeweiligen Aufgaben sollte dabei leicht ersichtlich sein. Mögliche Deadlines
oder Abgabefristen sollen bei jeder Aufgabe deutlich erkennbar sein.

Des Weiteren müssen Studierende die Möglichkeit haben, ohne die Installation von zusätzlichen Programmen die Aufgaben online bearbeiten, prüfen und abgeben
zu können. Trotzdem sollen sie die Option haben, die Aufgaben in der Entwicklungsumgebung ihrer Wahl lösen zu können.

Eine weitere wichtige Anforderung bei der Prüfung der Aufgaben ist, dass
die Bearbeiter der Aufgaben sogenannte
\glqq human-readable\grqq{}-Fehlermeldungen erhalten müssen. Das bedeutet, dass
die Fehlermeldungen bei der Überprüfung auch für nicht-technische Studenten
leicht verständlich sein müssen. Fehlermeldungen bzw. konstruktives Feedback
muss dabei automatisiert und jederzeit generiert werden können.

Die Möglichkeit Aufgabenversuche abzugeben, muss ebenfalls mit wenig Aufwand
behaftet sein. Falls ein Versuch fehlschlägt, oder nicht die volle Punktzahl
erhält, sollte der Student jederzeit einen neuen Versuch hochladen können.

\subsubsection{Lehrende}\label{anforderungsanalyse-funktional-lehrende}
Lehrende müssen neue Aufgaben anlegen und konfigurieren können, dazu gehört
unter anderem eine Deadline bzw. ein Abgabedatum. Des Weiteren sollen Lehrer
eine Übersicht an Aufgaben des jeweiligen Kurses haben. Außerdem sollten
Lehrende einzelne Aufgaben temporär verstecken oder deaktivieren können.

Überdies hinaus muss es möglich sein, mehrere Administratoren zu den Kursen
hinzuzufügen. Dadurch ist es möglich, dass verschiedene Personen die Aufgaben
erstellen und die abgegebenen Lösungen herunterladen können. Dies ist
gleichzeitig die nächste Anforderung: Administratoren müssen mit wenig
Aufwand alle Abgaben der Teilnehmer herunterladen können.

Ferner müssen die Aufgaben und die jeweiligen Deadlines auch nach der Erstellung
editierbar sein.

Lehrende müssen den Aufgaben-Fortschritt der Teilnehmer je nach Kurs 
übersichtlich einsehen können.
\subsection{Nichtfunktionale Anforderungen}
\label{anforderungsanalyse-nichtfunktional}
Das eingesetzte System muss neben den funktionalen Anforderungen auch diverse
nichtfunktionale Anforderungen erfüllen, um von der OTH als sinnvolle
Lernplattform eingesetzt werden zu können.

Der erste wichtige Punkt ist, dass die Lernplattform möglichst zuverlässig ist.
Zur Zuverlässigkeit gehört neben einer hohen Verfügbarkeit auch eine skalierende
Performance, wenn viele Studierende gleichzeitig die Plattform nutzen wollen.

Ein weiterer Punkt ist die Wartbarkeit. Die Plattform sollte mit möglichst wenig
Wartung sicher bestehen bleiben können. Außerdem sollte nur auf externe Systeme
gesetzt werden, von denen ausgegangen werden kann, dass diese noch einige Jahre
gepflegt werden. Hier empfiehlt sich ein Aufteilen der Plattform auf mehrere
externe Tools, um bei einem Ausfall oder Außerbetriebnahme eines einzelnen Tools
noch einen Notbetrieb gewährleisten zu können. Der Austausch gegen eine andere
neue Softwarekomponente gestaltet sich dadurch leichter.

Die Ressourcenlast und damit verbundenen Kosten spielen eine weitere wichtige
Rolle bei der Entscheidungsfindung. Wenn Teile der Lernplattform auf OTH-Servern
gehostet werden müssen, sollten diese möglichst ressourcenschonend sein.
Serverressourcen sind teuer und können das Projekt im Zweifelsfall unrentabel
machen, wenn das System bei paralleler Nutzung durch mehrere Studierende eine
inadäquate Serverlast voraussetzen würde. Selbiges gilt für Lizenzgebühren
möglicher Tools und Werkzeuge.

Neben den Kosten ist es auch wichtig, dass die Plattform zusammen mit dem
Learning Management System (LMS) bzw. der Lernplattform der Hochschule arbeiten
kann. Der Vorteil einer LMS-Integration wird später im Kapitel
\ref{code-freak} näher erläutert.

Zu guter Letzt ist es wünschenswert, dass der Programmierkurs mit der Auswahl
der Programmiersprache flexibel ist. So sollte es beispielsweise möglich sein,
dass die erste Aufgabe mit der Programmiersprache Java gelöst wird,
während die zweite mit Python gelöst werden muss.