\section{Zusammenfassung und Ausblick}\label{zusammenfassung-u-ausblick}
Nach der abgeschlossenen Studie folgt nun eine Zusammenfassung. Die
Aufgabenstellung war die Analyse, Konzeption und Implementierung eines Tools
für das automatisierte Testen, Einreichen und Benoten von Programmieraufgaben.
Diese Aufgabenstellung konnte nach erfolgreicher Analyse und Vergleich
bestehender Online-Programmierplattformen mithilfe von GitHub-Classroom als
Werkzeug erfolgreich absolviert werden.

Neben GitHub Classroom wird Replit als Online-Entwicklungsumgebung eingesetzt.
Um die Studierenden zu unterstützen, bekommen sie Zugang zu einem
GitHub-Repository, welches als Vorlage für die Einrichtung eines Workspace
dient. In diesem Replit Workspace können die Studierenden alle
Programmieraufgaben des Zusatzstudiums erledigen. Damit die Bearbeitung leichter
fällt, enthält das Template außerdem verschiedene Skripte. Darunter enthalten
sind Aufgabenmanagement-Tools, welche es erlauben, die Aufgaben herunterzuladen,
zu überprüfen und abzugeben, sowie ein Einrichtungsskript für GitHub und eine
eigene angepasste Konsoleninstanz. Die Aufgabenstellungen werden zusammen mit
den anderen Aufgaben des Kurses auf der Plattform Tutors gehostet.

Das Zusatzstudium Digital Skills kann durch den Einsatz dieser
Lernplattform sehr profitieren. Sowohl bei den Korrekturen, als auch bei den 
Aufgabenstellungen werden die zuständigen Dozierenden entlastet. Sie müssen
lediglich bereit sein, den Studierenden bei Fragen zur Seite zu stehen. Die
Materialien aller Fächer des Zusatzstudiums befinden sich auf der genannten
Plattform Tutors. Durch die Einbindung der Programmierplattform in die
vorhandenen Kursmaterialien wurde ein einheitliches Lernsystem geschaffen. Doch
auch die Teilnehmenden werden von der Programmierplattform profitieren. Alle
Module und Aufgaben von Digital Skills befinden sich sortiert an einem
gesammelten Ort. Dadurch müssen sich Studierende nicht in diversen Plattformen 
registrieren und können leichter die Übersicht behalten. Das selbstständige
Lösen der Aufgaben unterstützt außerdem den Lernprozess und intensiviert die
Lerninhalte.

Die durchgeführte Feldstudie war ein Erfolg. Schwierige Stellen und Hürden
konnten gefunden werden. Die Tests deckten außerdem auf, dass selbst einfache
Python-Aufgaben ohne exakt durchdachte Anleitungen für fachfremde Personen
schnell Probleme bereiten können.

Diese Kenntnisse können nun für den zukünftigen produktiven Einsatz im
Zusatzstudium Digital Skills berücksichtigt und angewandt werden. Die Ergebnisse
des Interviews zeigten auch, dass durchaus einige Studierende Interesse an dem
Zusatzstudium haben.

\newpage

Alles in allem hilft die in dieser Arbeit geplante, analysierte und
implementierte Online-Programmierplattform dem Zusatzstudium Digital Skills
dabei, die Digitalisierung in Deutschland voranzutreiben und zu stärken. Das
automatisierte Testen und Benoten von Programmieraufgaben unterstützt die
zuständige Abteilung ebenfalls in der Entlastung der Dozierenden. Den
technologischen Rückstand Deutschlands kann das Zusatzstudium sicherlich nicht
alleine aufholen. Trotzdem ist der Kurs Digital Skills zusammen mit der
dafür entwickelten Lösung, um online Programmieraufgaben lösen zu können, ein
wichtiger Schritt in die richtige Richtung. Das Zusatzstudium wirkt mit, um
zukünftige Generationen besser auf die digitale Welt und den Arbeitsplatz von
morgen vorzubereiten.
