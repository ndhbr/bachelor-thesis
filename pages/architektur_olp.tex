\subsection{Finale Architektur Online-Learning Platform}
\subsubsection{Tutors als Aufgabensammlung}
\subsubsection{GitHub Classroom als Abgabe- und Bewertungssystem}
GitHub Classroom eignet sich, für den Einsatz an der Hochschule, vor allem durch
seine Flexibilität und Einfachheit. Durch diese Flexibilität ist es möglich,
GitHub Classroom nur als eine austauschbare Komponente des Systems zu sehen.
Classroom übernimmt im System die Rolle des Aufgabenservers. Hier werden alle
Aufgabenvorlagen, sowie alle Versuche der Studierenden gespeichert und bewertet.

\subsubsection{Replit als Online-Entwicklungsumgebung}
Der Programmierkurs soll für jeden Teilnehmer ohne komplizierte Installationen
durchführbar sein. Aus diesem Grund wurde die Entscheidung gefällt, eine
Online Entwicklungsumgebung einzusetzen.

Der Vorteil an Replit ist, dass man dem Studierenden durch ein
Vorlagenrepository alle nötigen Konfigurationen und Programme im Vorhinein
bereitstellen kann. Sobald ein sogenanntes \glqq Repl\grqq{} (Bezeichnung für
ein Projekt in Replit) mit der erwähnten Vorlage erstellt wurde, kann der
Studierende ohne Installation geräteübergreifend online programmieren
\parencite{replit-import-from-github}. Der Teilnehmer muss daraufhin lediglich auf
den \glqq Run\grqq{}-Knopf drücken, welcher die später näher erläuterte
\glqq OTH-Console\grqq{} startet und schließlich alle benötigten Abhängigkeiten bereitstellt.