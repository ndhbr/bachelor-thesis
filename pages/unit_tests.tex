\subsection{Softwaretests}
\subsubsection{Allgemein}
Um Programmieraufgaben vollständig automatisiert bewerten zu können, benötigt
man eine Methode, welche es ermöglicht einzelne Funktionen oder Klassen eines
Quellcodes auf ihre Funktionalität zu testen. In professionellen Softwareteams
kommen oft \emph{Unit-Tests}, oder auch Modultests bzw. Komponententests zum
Einsatz. Sie dienen dazu, mit einer gewissen \emph{Code Coverage}
(Testabdeckung) sicherstellen zu können, dass alle Komponenten einer Applikation
ordnungsgemäß funktionieren.

\subsubsection{Unit-Testing}\label{unit-tests}
Unit-Testing ist eine Methode, um Software-Quellcode zu testen. Bei dieser
Methode werden einzelne Komponenten bzw. Einheiten (Units) des Programms
unabhängig voneinander mit allen erforderlichen Abhängigkeiten getestet. Meist
werden Unit-Tests von den Entwicklern geschrieben, die auch das jeweilige
Modul des Programms entwickelt haben. Im Fall des Online-Programmierkurses
werden die Unit-Tests durch die Aufgabensteller erstellt. Der Studierende kann
dann seinen Lösungsversuch der jeweiligen Übung mithilfe der bereits gegebenen
Komponententests (Unit-Tests) überprüfen. \parencite{unit-test}

In folgendem Beispiel wird mithilfe der Python Unit-Test-Bibliothek Pytest
ein einfacher Komponententest durchgeführt:

\begin{lstlisting}[style=Python]
    # test_my_name_is.py

    # Zu testende Funktion
    def my_name_is(name):
        return f'Hallo, mein Name ist {name}'

    # Unit-Test Funktion
    def test_my_name_is():
        assert my_name_is('Andreas') == 'Hallo, mein Name ist Andreas'
\end{lstlisting}

In diesem Code-Ausschnitt wird die Methode \texttt{my\_name\_is(name)} mithilfe
der Unit-Test-Funktion \code{test\_my\_name\_is()} getestet. Die Methode 
\code{my\_name\_is(name)} in Zeile 4 hat die Aufgabe, abhängig des in der
Variable \code{name} übergebenen Namens, die Nachricht \glqq Hallo, mein Name
ist \code{<NAME>}\grqq{} auszugeben. \code{<NAME>} ist hier ein Platzhalter
die Variable \code{name}.

Die Unit-Test-Funktion \code{test\_my\_name\_is()} ruft die Methode
\texttt{my\_name\_is(name)} mit dem Namen \glqq Andreas\grqq{} auf und
vergleicht das Ergebnis der Funktion mit dem erwarteten Ergebnis. Entscheidend
für diesen Vergleich ist das Keyword \code{assert}. Nun kann der Test mit
Pytest gestartet werden. Hierfür benötigt man lediglich folgenden
Kommandozeilen-Befehl:

\begin{lstlisting}[style=Bash]
    $ pytest test_my_name_is.py
\end{lstlisting}

Daraufhin erhält man beim erfolgreichen Bestehen des Tests folgende Ausgabe:

\begin{lstlisting}[style=Bash]
collected 1 item                                                               

test_my_name_is.py .                                                                  [100%]

================================ 1 passed in 0.00s =================================
\end{lstlisting}

Würde man die Methode \texttt{my\_name\_is(name)} verändern und in das
\glqq Hallo\grqq{} ein weiteres \glqq a\grqq{} einsetzen, würde der
Komponententest wie folgt aussehen:

\begin{lstlisting}[style=Bash]
collected 1 item                                                                                                                      

test_my_name_is.py F                                                                  [100%]

===================================== FAILURES =====================================
__________________________________ test_my_name_is _________________________________

    def test_my_name_is():
>     assert my_name_is('Andreas') == 'Hallo, mein Name ist Andreas'
E     AssertionError: assert 'Haallo, mein...e ist Andreas' == 'Hallo, mein Name ist Andreas'
E       - Hallo, mein Name ist Andreas
E       + Haallo, mein Name ist Andreas
E       ?  +

test_my_name_is.py:6: AssertionError
============================= short test summary info =============================
FAILED test_my_name_is.py::test_my_name_is - AssertionError: assert 'Haallo, mein...e ist Andreas' == 'Hallo, mein Name ist Andreas'
================================= 1 failed in 0.01s ================================
\end{lstlisting}

Pytest schlüsselt den Fehler sehr genau auf und visualisiert, wie in der Ausgabe
zu erkennen ist, die Fehlerstelle auf das Zeichen genau. Direkt unter die
erwartete Ausgabe gibt Pytest die tatsächliche Ausgabe aus. So kann man den
entstandenen Fehler sehr schnell finden und beheben.

Die Genauigkeit und Lesbarkeit der Ausgabe ist wichtig, damit die Studierenden
später bei den Programmierübungen wenig bis keine Schwierigkeiten bei der
Bearbeitung haben.