\section{Studie: Test an fachfremden Studierenden}\label{studie}
\subsection{Allgemeines}
In einer Studie sollen Fehler und mögliche Hindernisse in den Aufgaben der
Programmierplattform gefunden und analysiert werden. Damit die Studie möglichst
realitätsnah ist, fällt die Wahl hierbei auf eine Feldstudie mit fachfremden
Studierenden. Wie im späteren Anwendungsfall auch werden Studierende mithilfe
der Online-Programmierplattform selbstständig Aufgaben lösen.
\parencite{feldstudie}

Nach Vollendung der Tests sollen Schwachstellen und Lücken der
Programmieraufgaben (Anleitungen, Fehlerbeschreibungen, Aufbau, Schwierigkeit,
...) gefunden, behoben und für zukünftige Aufgaben berücksichtigt werden.

\subsection{Methode}
Um für jeden Durchlauf gleiche Testbedingungen sicherstellen zu können, wird von
Herrn Huber ein Testkonzept ausgearbeitet. Bei jeder Durchführung wird sich auf
die Regeln und den Ablauf des Testkonzepts bezogen. Das vollständige Testkonzept
befindet sich im Anhang \ref{appendix:testkonzept}.

Das Testkonzept legt im Grunde die Metadaten der Studie fest. Neben der
Zielsetzung, der Zielgruppe und der Dauer des Tests wird in dem Konzept auch die
Methodik festgelegt.

\subsubsection{Teilnehmende}
Die Zielgruppe der Studie lässt sich durch wenige Parameter definieren. Sie
ist äquivalent zur Zielgruppe des Zusatzstudiums Digital Skills. Gesucht sind
folglich Studierende, welche keine Studiengänge der Fakultät Informatik und
Mathematik belegen. Mit dieser Vorraussetzung qualifiziert man sich für die
Teilnahme am Zusatzstudium Digital Skills und dadurch implizit auch für die
Anteilnahme an der folgenden Studie.

Um eine gleichmäßig verteilte Stichprobenmenge zu erhalten, werden mindestens
fünf Testpersonen aus fünf unterschiedlichen Studiengängen für die Durchführung
benötigt. Wie auf der nächsten Seite in \autoref{table:studie-teilnehmende}
absolut dargestellt, ist das Geschlechterverhältnis mit 40 \% männlichen
Teilnehmern sehr ausgeglichen. Folgende Tabelle enthält neben dem Geschlecht
auch das jeweilige Alter, sowie den Studiengang und das Semester der
Testpersonen. Das Alter ist wichtig, um eine realistische Abbildung der
Zielgruppe von Digital Skills zu erzeugen.

\begin{table}[H]
\renewcommand*{\arraystretch}{1.6}
\centering
\begin{tabular}{|l|l|l|l|}
\hline
\textbf{Geschlecht} & \textbf{Alter} & \textbf{Studiengang}                      & \textbf{Semester} \\ \hline
Weiblich            & 22             & Betriebswirtschaftslehre (B.A.)           & 5                 \\ \hline
Weiblich            & 21             & Bauingenieurwesen (B.Eng.)                & 5                 \\ \hline
Weiblich            & 22             & Psychologie (M.Sc.)                       & 3                 \\ \hline
Männlich            & 23             & Brauwesen und Getränketechnologie (M.Sc.) & 1                 \\ \hline
Männlich            & 23             & Regenerative Energietechnik (B.Eng.)     & 6                 \\ \hline
\end{tabular}
\caption{Studienteilnehmer}

\label{table:studie-teilnehmende}
\end{table}

\subsubsection{Ablauf des Tests}
Die Studienbefragungen fanden zwischen dem 11. Februar 2022 und dem 6. März 2022
statt. Die Tests wurden in Form eines Interviews online über eine Software für
Videokonferenzen durchgeführt.

Am Anfang wurde vom Beobachter der Studie, Andreas Huber, das Zusatzstudium
Digital Skills anhand eines einseitigen Infodokuments vorgestellt (siehe Anhang
\ref{appendix:dscc-info}). Das Infodokument wurde von Frau Lachmann erstellt und
enthält eine Kurzzusammenfassung des Programms. Frau Lachmann ist zudem 
zuständig für die Konzeption und Evaluation innerhalb des Zusatzstudiums.

Danach wurde den Teilnehmern die Agenda (siehe Anhang
\ref{appendix:testkonzept}) vorgetragen. Anschließend wurden den Personen
jeweils vier Fragen gestellt. Die Fragen mussten mit Schulnoten von 1 bis 6
beantwortet werden. Die Note 6 bedeutete hier immer eine vollständige
Verneinung, während eins einer völligen Zustimmung entsprach. Alle restlichen
Noten bildeten wie gewohnt lineare Zwischenwerte.

Nachdem die Fragen von der teilnehmenden Person beantwortet wurden, wurde sie
dazu aufgefordert die Bildschirmübertragung starten. Währenddessen notierte der
Beobachter die Ergebnisse der Fragen in eine vorher angelegte Tabelle.

Der Teilnehmer wurde nachfolgend darauf hingewiesen, dass die Studie der
Think-Aloud-Methodik folgt. Die Think-Aloud-Methode ist eine Forschungsmethode,
bei der der Testteilnehmer darum gebeten wird, seine Gedanken laut
auszusprechen. Mit dieser Methode können mögliche Aufhänger und Probleme in den
Anleitungen der Aufgaben leichter gefunden werden. Der Beobachter des Tests
schreibt alle Anomalien kategorisiert nach Teilnehmenden und Fortschritt des
Tests auf. \parencite{think-aloud}

Aufkommende Fragen konnten vom Beobachter beantwortet werden. Sollte eine Frage
bzw. Aufgabe unlösbar erscheinen, gab es am Ende jeder Aufgabe einen
Lösungsvorschlag. Dieser sollte jedoch nur aufgeklappt werden, wenn keine
realistische Chance der selbstständigen Lösung des Problems bestand.

Als Nächstes wurden die im folgenden \autoref{studie-materialien} erläuterten
Aufgaben durch den Testteilnehmer bearbeitet. Während der Bearbeitung wurde die
dafür benötigte Zeit durch den Beobachter mit einer Stoppuhr gestoppt.

Abschließend konnte der Studierende die Bildschirmübertragung wieder beenden und
sich auf das Nachgespräch konzentrieren. Hierbei wurden dem Teilnehmer noch
fünf weitere Fragen gestellt. Die Fragen mussten -- wie vorher auch -- in
Schulnoten von 1 bis 6 beantwortet werden. Die Ergebnisse der Fragen wurden vom
Beobachter notiert und in die vorher erwähnte Datentabelle eingetragen.

Als grober Leitfaden wurde eine Dauer von 30 bis 60 Minuten festgelegt, dies
sollte jedoch kein hartes Zeitlimit darstellen. Der Test sollte möglichst
unabhängig ablaufen und konnte bei möglichen Unverständnissen auch deutlich
länger dauern. Gemäß einer Feldstudie darf das Ergebnis der Studie nicht durch
eine mögliche  Manipulation, wie beispielsweise durch eine zeitliche Barriere,
verfälscht werden.

Auf eine Entschädigung der Teilnehmenden wurde aufgrund mangelnder Budgets
verzichtet. Alle Teilnehmenden absolvierten die Studie freiwillig.

\subsubsection{Materialien}\label{studie-materialien}
\paragraph{Fragen}
Folgende Fragen wurden den Teilnehmenden vor Beginn der Bearbeitung gestellt:

\begin{itemize}
    \item Hast du schon einmal erwägt, eine Programmierausbildung anzustreben?
    (1: will ich definitiv noch machen; 6: noch nie) \textbf{[PRE1]}
    \item Hast du schon einmal programmiert? (1: ständig; 6: noch nie)
    \textbf{[PRE2]}
    \item Wie fit fühlst du dich am PC? (1: sehr fit; 6: gar nicht)
    \textbf{[PRE3]}
    \item Würde für dich das Zusatzstudium Digital Skills in Frage kommen?
    (1: unbedingt; 6: auf keinen Fall) \textbf{[PRE4]}
\end{itemize}

Nach Bearbeitung der Aufgaben mussten die Studierenden noch folgende fünf
Fragen beantworten:

\begin{itemize}
    \item Wie schwer kam dir die Einrichtung, bis zum Zeitpunkt, an dem du die
    erste Aufgabe heruntergeladen hast, vor? (1: sehr leicht; 6: sehr schwer)
    \textbf{[PAST1]}
    \item Hattest du Schwierigkeiten mit Aufgabe 1? (1: nein, keine; 6: zu
    komplex) \textbf{[PAST2]}
    \item Hattest du Schwierigkeiten mit Aufgabe 2? (1: nein, keine; 6: zu
    komplex) \textbf{[PAST3]}
    \item Hast du verstanden, was du in den Aufgaben genau gemacht hast? (1: ja
    vollkommen; 6: nein, gar nicht) \textbf{[PAST4]}
    \item Würdest du nach Abschluss des Tests deine Meinung zur Frage am
    Interesse eines Zusatzstudiums für Digital Skills ändern? (1: unbedingt;
    6: auf keinen Fall) \textbf{[PAST5]}
\end{itemize}

Alle aufgezeigten Fragen wurden durch den Beobachter und Initiator der Studie,
Andreas Huber, selbst entwickelt. Die in eckigen Klammern stehenden
Bezeichnungen dienen zur Identifikation der jeweiligen Fragen für die später
folgende Auswertung.

\paragraph{Aufgaben}
Der Test besteht aus vier für die Studie vereinfachten Aufgaben. Die erste
Aufgabe beschäftigt sich lediglich mit der Einrichtung des Arbeitsplatzes in
Replit und GitHub-Classroom. Die Studierenden sind dazu aufgefordert, das in 
\autoref{replit-template} besprochene Replit-Template zu klonen und in  Replit
einzurichten. Dazu gehört das Hinzufügen des SSH-Keys zu GitHub, sowie das
Herunterladen der ersten Aufgabe. \parencite{git-repo:replit-template}

Die zweite Aufgabe ist eine Programmieraufgabe mit der Sprache Python. Es
handelt sich hierbei um eine vereinfachte Version der später in produktiv
eingesetzten Aufgabe \glqq Lab 4: Hello\grqq{}. Die bereits gegebene Vorlage der
Aufgabe fragt den User nach seinem Namen. Nach der Eingabe schließt sich das
Programm wieder. \parencite{git-repo:first-loop}

Die Aufgabe des Teilnehmenden ist es nun den Namen in folgenden
Format wieder auszugeben: \code{Hallo, mein Name ist <NAME>}. Der Platzhalter
\code{<NAME>} soll dabei durch den vorher abgefragten Namen ersetzt werden.
Diese Aufgabe kann der Teilnehmer mit einer einzigen Codezeile lösen. Die
Ausgabe der Lösung sieht, wie auf der nächsten Seite dokumentiert, in etwa so
aus:

\begin{lstlisting}[style=Bash]
$ python main.py
Wie ist dein Name? Andreas
Hallo, mein Name ist Andreas!
\end{lstlisting}

Der ausgegebene Name ist wie vorher beschrieben flexibel und abhängig von der
Eingabe des Nutzers.

Die dritte Aufgabe ist ebenfalls eine Programmieraufgabe mit der Sprache Python.
In dieser Übung geht es um die erste Verwendung einer \texttt{for}-Schleife.
Am Anfang der Anleitung wird sehr ausführlich erklärt, was eine
\texttt{for}-Schleife ist, weshalb man sie braucht und wie man sie anwendet.
Danach wird die Aufgabenstellung erklärt, welche ähnlich zur zweiten Aufgabe
ist. \parencite{git-repo:first-loop}

Dieses Mal wird der Nutzer nach Start des Programms nicht nur nach seinem
Namen gefragt, sondern auch nach der Anzahl, wie oft der Name ausgegeben werden
soll. Wie auch in der vorherigen Aufgabe schließt sich standardmäßig das
Python-Skript gleich wieder. Der Testteilnehmer muss nun eine Schleife
programmieren, sodass der Name mit der aktuellen Zählvariable n-mal ausgegeben
wird. Der Name und die Anzahl an Ausgaben wird wieder per Funktion als Parameter
übergeben. Die Ausgabe in der Konsole soll wie folgt aussehen:

\begin{lstlisting}[style=Bash]
$ python main.py
Wie ist dein Name? Andreas
Wie oft soll der Name ausgegeben werden? 5
Andreas 0
Andreas 1
Andreas 2
Andreas 3
Andreas 4
\end{lstlisting}

\newpage

Falls die Aufgaben reibungslos funktioniert haben, kann der Studierende
die Herausforderung einer optionalen Bonusaufgabe annehmen. Um diese
Bonusaufgabe zu lösen, muss der Teilnehmer die vorherige Aufgabe modifizieren,
sodass die Ausgabe in der Konsole wie folgt aussieht:

\begin{lstlisting}[style=Bash]
$ python main.py
Wie ist dein Name? Andreas
Wie oft soll der Name ausgegeben werden? 5
Andreas 1
Andreas 2
Andreas 3
Andreas 4
Andreas 5
\end{lstlisting}

Der Unterschied hier ist die Nummerierung. Im Standardfall zählt das Programm
von 0 bis n - 1. Um die Bonusaufgabe zu bewältigen, muss das Programm von 1 bis
n zählen.

Die vierte und letzte Aufgabe beschäftigt sich mit der Generierung von
Text-Pyramiden, welche der ursprünglichen Version des Videospiels Super Mario
Bros.\copyright\footnote{Super Mario Bros.\copyright\ ist eine geschützte Marke
der Nintendo Co., Ltd.} entsprechen sollen. \parencite{git-repo:mario-less}

Der Nutzer wird beim Start des Programms nach der gewünschten Größe bzw. Höhe
der Pyramide gefragt. Nach Eingabe eines gültigen Werts liefert das Skript,
ausgegeben mit Raute-Zeichen, die linke Seite einer klassischen Super Mario
Pyramide. Die Ausgabe der gelösten Aufgabe sieht beispielhaft so aus:

\begin{lstlisting}[style=Bash]
$ python main.py
Höhe: -1
Höhe: 0
Höhe: 6
     #
    ##
   ###
  ####
 #####
######
\end{lstlisting}

\newpage

Das Beispiel zeigt neben der Pyramide auch, dass das Skript ungültige
Größenangaben erkennen und ignorieren soll. Aufgrund der erhöhten Schwierigkeit
dieser Aufgabe wurde entschieden, die letzte Aufgabe für die Feldstudie zu
verwerfen.

\paragraph{Test-Classroom}
Für die Vorbereitung der Interviews wird eine neue GitHub-Organisation, sowie
ein neuer GitHub-Classroom angelegt. Sowohl das Replit-Template als auch die im
vorherigen Kapitel beschriebenen Aufgaben werden als Template-Repositorys in
der Test-Organisation abgelegt. \parencite{git-orga:ndhbr-classroom}

Für die Aufgaben werden mithilfe des Test-Frameworks Pytest Benotungstests
programmiert und in GitHub-Classroom konfiguriert. Jede gelöste Aufgabe belohnt
den Studienteilnehmenden pauschal mit zehn Punkten.

Neben den Programmieraufgaben werden auch ausführliche Anleitungen mit Bildern
geschrieben. Diese befinden sich in den jeweiligen Aufgabenrepositorys unter dem
Dateinamen \texttt{README.md}. Nach der Erstellung von Vorschaubildern können
diese in der, auch später produktiv genutzten, Plattform Tutors hochgeladen
werden. Die Studienteilnehmenden erhalten zu Beginn des Probelaufs einen Link
zur Übersicht der benötigten Anleitungen in Tutors.

\newpage
\subsection{Deskriptive Ergebnisse}
Nach Verarbeitung der Ergebnisse der Fragen können nun die einzelnen Resultate
analysiert und interpretiert werden. Die folgende \autoref{table:studie} enthält
in der ersten Spalte die Identifikationsbezeichnung der Frage. Die zweite und
dritte Spalte beinhaltet den Durchschnitt und die dazugehörige \ac{sd}. Die
vierte, fünfte und sechste Spalte enthalten den Medianwert, das Minimum und
schließlich das Maximum der Antworten.

\begin{table}[H]
    \renewcommand*{\arraystretch}{1.6}
    \centering
    \begin{tabular}{|l|l|l|l|l|l|} 
    \hline
    \diagbox{\textbf{Fragen}}{\textbf{Ergebnisse}} & \textbf{Durchschnitt } & \textbf{SD} & \textbf{Median } & \textbf{Min.} & \textbf{Max.}  \\ 
    \hline
    \textbf{Alter }                                & 22.200                 & 0.837       & 22.000           & 21            & 23             \\ 
    \hline
    \textbf{PRE1 }                                 & 3.600                  & 2.300       & 3.000            & 1             & 6              \\ 
    \hline
    \textbf{PRE2 }                                 & 2.600                  & 1.670       & 3.000            & 1             & 5              \\ 
    \hline
    \textbf{PRE3 }                                 & 2.600                  & 0.894       & 2.000            & 2             & 4              \\ 
    \hline
    \textbf{PRE4 }                                 & 2.600                  & 1.820       & 2.000            & 1             & 5              \\ 
    \hline
    \textbf{PAST1 }                                & 2.600                  & 1.520       & 2.000            & 1             & 5              \\ 
    \hline
    \textbf{PAST2 }                                & 3.200                  & 1.300       & 4.000            & 1             & 4              \\ 
    \hline
    \textbf{PAST3 }                                & 3.800                  & 1.790       & 5.000            & 1             & 5              \\ 
    \hline
    \textbf{PAST4 }                                & 2.200                  & 1.100       & 3.000            & 1             & 3              \\ 
    \hline
    \textbf{PAST5 }                                & 3.800                  & 2.170       & 3.000            & 1             & 6              \\
    \hline
    \end{tabular}

    \caption{Auswertung der Studienergebnisse}
    \label{table:studie}
\end{table}

% Alter
Wie man der Alterszeile entnehmen kann, sind die Teilnehmenden im Schnitt 22
Jahre alt. Das Durchschnittsalter trifft genau den deutschen Altersdurchschnitt
von Studierenden und ist somit als Zielgruppe für das Zusatzstudium geeignet.
\parencite{studie-alter-studierenden}

% PRE1, PRE2
Die Ergebnisse der Fragen PRE1 und PRE2 deuten darauf hin, dass das Interesse
an der Informatik sehr gemischt sind. Der Bereich zwischen Minimum und Maximum
ist in beiden Fragen sehr hoch. Dies verdeutlicht, dass es sowohl Teilnehmende
gibt, die bereits ständig mit Programmieren zu tun haben, als auch welche, die
noch absolut keine Berührungen mit der Materie hatten.

% PRE3
Das Wohlbefinden am Computer wurde mit der Frage PRE3 abgefragt. Ein
Interpretationsversuch der Minima und Maxima ist, dass die Teilnehmenden aus
Angst später Fehler zu machen niemals die Note 1 vergeben haben. Damit ist diese
Frage auch die einzige mit Schulnoten zu bewertende Frage, welche in der Umfrage
keine Note 1 als Minimum erhalten hat.

% PRE4 - PAST5
Die Frage PRE4 bezieht sich auf das Interesse an einem Zusatzstudium für
Digital Skills. Der Durchschnitt der Antworten auf diese Frage liegt bei 2,600,
der Median noch besser bei 2,000. Das heißt, dass die Mehrheit der Testpersonen
vor der Bearbeitung der Aufgaben, einer Teilnahme an einem solchen Zusatzstudium
nicht abgeneigt waren. Nach der Absolvierung der Programmieraufgaben werden die 
Teilnehmer erneut nach ihrer Meinung bzw. einer Änderung dieser Einschätzung
gefragt (Frage: PAST5). Hier wird anhand des Durchschnitts, sowie des Medians
eine deutliche negative Tendenz ersichtlich. Durch die hohe Standardabweichung
bei der Frage PAST5, kann man nicht auf eine vollständige Abneigung der
Studierenden schließen. Trotzdem kann die negative Tendenz des Interesses auf zu
schwere oder dürftig erklärte Aufgaben hindeuten. Ein weiterer
Interpretationsversuch ist, dass die Studierenden nach der Präsentation des 
Zusatzstudiums andere Vorstellungen zum Programmiermodul hatten.

% PAST1, PAST2, PAST3
Die Fragen PAST1, PAST2 und PAST3 beziehen sich jeweils auf das
Schwierigkeitsempfinden der Studierenden in Bezug auf die drei
Programmieraufgaben. Die erste Programmieraufgabe ist in diesem Fall die
Einrichtungsaufgabe des Arbeitsbereichs. Mit dem Design der Aufgaben wird
versucht die Schwierigkeit mit jeder Aufgabe zu steigern. Die Durchschnitte der
Umfrage bestätigen den Erfolg eines linearen Schwierigkeitsanstiegs. Während
PAST1 einen Durchschnitt von 2,600 aufweist, scheint die Schwierigkeit bei der
ersten Python-Programmieraufgabe (PAST2) mit einem Durchschnittswert von 3,200
deutlich höher zu liegen. Erneut gesteigert hat sich das Schwierigkeitsempfinden
mit der dritten Aufgabe (PAST3) auf einen Durchschnittswert von 3,800. Die
Differenz der Durchschnitte beträgt bei beiden Anstiegen genau 0,800
Notenpunkte.

% PAST4
Das allgemeine Verständnis der Aufgaben wird mit der Frage PAST4 abgefragt. Für
diese Frage wurde die 3 als Maximalnote aufgezeichnet. Der Durchschnitt liegt
wiederum bei 2,200 Notenpunkte und ist somit im positiven Bereich.

% Allgemeine Notizen
Die Auswertung der handschriftlichen Notizen des Studienbeobachters haben
ergeben, dass es vor allem wichtig ist Variablen, Methodenparameter und
Kontrollstrukturen vorher genau zu erklären (siehe Anhang
\ref{appendix:feldstudie-notizen}). Für fachfremde Personen erschien das Konzept
dahinter anfangs sehr verwirrend und überfordernd.

Außerdem wurde festgestellt, dass zwei Anweisungen in einer Zeile meist dazu
führten, dass nach Befolgung der ersten Anweisung die zweite übersprungen wurde
und direkt mit der nächsten Zeile fortgefahren worden ist. Daraus folgt die
Erkenntnis, dass maximal eine Anweisung pro Zeile gegeben werden sollte.

Zu guter Letzt taten sich nach Angaben der Notizen viele Studierende schwer,
einen Überblick über die offenen Dokumente und Browsertabs zu halten. Die
Erklärungen zu den in der OTH-Console installierten Hilfsprogramme wurden meist
bis zum Bedarf bei der Bearbeitung der Aufgaben wieder vergessen. Hier würden
möglicherweise stellenweise Wiederholungen der Befehle an den richtigen Stellen
weiterhelfen. Ein Inhaltsverzeichnis mit genauen Links zu den einzelnen Kapiteln
und Aufgaben würde Interessierten außerdem helfen den Überblick zu behalten.
