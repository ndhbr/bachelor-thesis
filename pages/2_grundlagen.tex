\section{Toolchain}
Dieses Kapitel befasst sich mit der für das Projekt benötigten Toolchain.
Eine Toolchain ist eine Sammlung verschiedener Anwendungen, die gemeinsam eine
Lösung bzw. ein Produkt erzeugen. Durch den Vergleich mit verschiedenen
bestehenden digitalen Lernplattformen ist es möglich, optimale 
Software-Werkzeuge für die Hochschule Regensburg zu finden und schließlich
einzusetzen.

\subsection{Vergleich vorhandener Systeme}
\subsubsection{CS50 der Harvard University}
\paragraph{Allgemeines}
CS50 ist die ursprüngliche Bezeichnung eines Lernkurses über Informatik,
welcher von der Harvard University ins Leben gerufen wurde und weiterhin
betreut wird. Der Kurs wurde aufgrund seines Erfolgs digitalisiert und wird nun
als CS50x auf der Lernplattform edX angeboten. Folgende Recherchen und Aussagen
sind jeweils immer auf die Online-Version CS50x bezogen.

Der Kurs CS50 lehrt Schüler die Grundlagen der Informatik. Dabei werden
diverse Programmierübungen abgefragt. Aufgrund der hohen Anzahl an Teilnehmern
besitzt der Kurs ein automatisiertes Abgabe- und Benotungssystem.

Das System hinter CS50 wird mittlerweile vielfältig eingesetzt und wurde zu
einem universalen Online-Lernsystem erweitert. Jeder kann sich durch eine
Authentifizierung über die Plattform GitHub im Abgabesystem von CS50 einloggen
und eigene Kurse erstellen.

\paragraph{Ablauf}
Dem Teilnehmer wird jede Woche ein neues Kapitel präsentiert. Er kann sich dabei
sowohl durch ein Vorlesungsvideo, als auch durch geschriebene Materialien über
das Thema der Woche informieren. Mit Beginn der Woche bekommt der Teilnehmer
neben den Materialien auch Programmieraufgaben, welche er mit dem vorher
genannten System bearbeiten kann.

Die Programmieraufgaben können wahlweise über die, auf AWS Cloud9 basierenden,
Online-Entwicklungsumgebung \glqq CS50-IDE\grqq{} von Harvard oder in jeder
anderen beliebigen Entwicklungsumgebung der Wahl bearbeitet werden. Dies
wird durch die Architektur des Systems ermöglicht. Jede Funktionalität der
Automatisierung geschieht durch Kommandozeilen-Tools. Dieses System hat den
Vorteil, dass es unabhängig von der eingesetzten IDE funktioniert, es wird
lediglich ein Terminal mit den jeweiligen Tools benötigt.

Vor der Abgabe der endgültigen Lösung mit dem sogenannten Werkzeug
\glqq submit50\grqq{}, ist es möglich den Code mit einem weiteren Werkzeug
namens \glqq check50\grqq{} überprüfen zu lassen. Außerdem gibt es viele weitere
Werkzeuge, ein Beispiel hierfür ist \glqq style50\grqq{}, welches die Qualität
und den Style des Programmcodes überprüft und bewertet.

\paragraph{Architektur}
Die Harvard University hält den Aufbau von CS50 weitestgehend transparent.
Viele der eingesetzten Werkzeuge sind öffentlich als Open-Source-Projekte unter
der GitHub-Organisation \glqq CS50\grqq{} zu finden. Darunter befinden sich
unter Anderem folgende Projekte:
\begin{itemize}
\item submit50: Abgabe von Code
\item check50: Funktionalitätstests des Codes
\item render50: Erzeugung von .PDF-Dateien aus Code
\item ide50: Online-Entwicklungsumgebung
\item style50: Überprüfung der Code-Qualität
\item compare50: Plagiatserkennung von abgegebenen Projekten
\item server50: Webserver
\end{itemize}

// TODO: Bild der Architektur mit Erklärung

\paragraph{Probleme beim Einsatz für die OTH}
Die Toolchain von CS50 wäre adäquat für den Einsatz an der OTH-Regensburg.
Eines der Projekte ist aktuell noch nicht Open-Source: Die Website zur
Erstellung von neuen Kursen, Abgaben und Mitgliederverwaltung. Dieses Projekt
ist essentiell für die Verwendung der Werkzeuge an der Regensburger Hochschule.
Nach Rücksprache mit Herr X der Harvard University ist ein Neuaufbau dieser
Website mit einhergehender Veröffentlichung als Open-Source-Projekt gerade in
Planung. Einen genauen öffentlichen Zeitplan hierfür gibt es aktuell nicht. In
Folge dessen ist ein Einsatz des CS50-Systems an der OTH zum heutigen Datum
nicht möglich.

\subsubsection{Code FREAK der Universität Kiel}
\subsubsection{nbgrader als Plugin für Jupyter Notebooks}
