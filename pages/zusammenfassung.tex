\section{Zusammenfassung und Ausblick}\label{zusammenfassung-u-ausblick}
Nach erfolgreicher Analyse und Vergleich bestehender
Online-Programmierplattformen wird GitHub-Classroom als Werkzeug zum
automatisierten Testen, Einreichen und Benoten von Programmieraufgaben
ausgewählt.

Neben GitHub Classroom wird Replit als Online-Entwicklungsumgebung eingesetzt.
Um die Studierenden zu unterstützen, bekommen sie Zugang zu einem
GitHub-Repository, welches als Vorlage für die Einrichtung eines Workspaces
dient. In diesem Replit Workspace können die Studierenden alle
Programmieraufgaben des Zusatzstudiums erledigen. Damit die Bearbeitung leichter
fällt, enthält das Template außerdem verschiedene Skripte. Darunter enthalten
sind Aufgabenmanagement-Tools, welche es erlauben die Aufgaben herunterzuladen,
zu überprüfen und abzugeben, sowie ein Einrichtungsskript für GitHub und eine
eigene angepasste Konsoleinstanz. Die Aufgabenstellungen werden zusammen mit den
anderen Aufgaben des Kurses auf der Plattform Tutors gehosted.

Das Zusatzstudium Digital Skills profitiert durch den Einsatz dieser
Lernplattform sehr. Sowohl bei den Korrekturen, als auch bei den 
Aufgabenstellungen werden die zuständigen Dozierenden entlastet. Sie müssen
lediglich bereit sein, wenn Studierende bei der Bearbeitung der Übungen Fragen
haben, zu helfen. Doch auch die Teilnehmenden werden von der
Programmierplattform profitieren. Das selbstständige Lösen der Aufgaben
unterstützt den Lernprozess.

Die durchgeführte Feldstudie war ein Erfolg. Schwierige Stellen und Hürden
konnten gefunden werden und können nun für den produktiven Einsatz der Plattform
angepasst werden. Die Ergebnisse des Interviews zeigten, dass durchaus einige
Teilnehmenden Interesse an dem Zusatzstudium haben. Außerdem deckte die Studie
auf, dass selbst einfache Python-Aufgaben ohne exakt durchdachte Anleitungen
für fachfremde Personen schnell Probleme bereiten können.

Den technologischen Rückstand Deutschlands kann das Zusatzstudium sicherlich
nicht alleine aufholen. Trotzdem ist der Kurs Digital Skills zusammen mit der
dafür entwickelten Lösung, um online Programmieraufgaben lösen zu können, ein
guter Schritt in die richtige Richtung. Das Zusatzstudium hilft zukünftige
Generationen besser auf die digitale Welt vorzubereiten.