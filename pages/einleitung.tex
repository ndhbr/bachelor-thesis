\section{Einleitung und Motivation}\label{einleitung}
\subsection{Herausforderung Digitales Lernen}\label{herausforderung}
Deutschland gilt als sehr rückschrittlich im Thema Digitalisierung an Schulen
und Universitäten. Nicht zuletzt hat die Corona-Pandemie den Rückstand
Deutschlands in vielen Bereichen offengelegt. Neben Verwaltungen, Unternehmen,
Schulen oder Gerichten hat die Pandemie vor allem viele Hochschulen dazu
gezwungen, umzudenken und digitale Lerninhalte bereitzustellen. Mit dieser
Herausforderung kamen jedoch einige Einrichtungen und Lehrende\footnote{Aus
Gründen der besseren Lesbarkeit wird in der gesamten Arbeit auf die
gleichzeitige Verwendung der Sprachformen männlich, weiblich und divers (m/w/d)
verzichtet. Sämtliche Personenbezeichnungen gelten gleichermaßen für alle
Geschlechter. Ist eine spezifische Geschlechtergruppe gemeint, wird das
entsprechende Adjektiv vorangestellt (z.B. \glqq männliche Studenten\grqq)} ohne
jegliche Vorbereitungen schnell an ihre Grenzen. \parencite{bmwi-rueckstand}

Professor Doktor Markus Heckner hat sich dieser Problemstellung angenommen und
hat gemeinsam mit Mitarbeitenden der Hochschule Regensburg an einer Lösung
gearbeitet. Das Ergebnis ist ein optionales Zusatzstudium, um die sogenannten
\glqq Digital Skills\grqq{} der Studierenden zu verbessern und zu intensivieren.
Langfristig gesehen sollen die Zusatzmodule die digitalen Fähigkeiten der
Teilnehmenden fördern und somit zur Einholung des digitalen Rückstands in
Deutschland beitragen. \parencite{digital-skills}

\subsection{Kurs Digital Skills als Zusatzstudium}\label{kurs-digital-skills}
Digital Skills ist ein aus drei Semestern bestehendes Zusatzstudium für alle
Studierenden der Hochschule Regensburg. Ausgenommen hiervon sind Studierende der
Fakultät Informatik und Mathematik. Wie der Name bereits sagt, findet das
Programm parallel zum Hauptstudium der Studierenden statt.

Das Zusatzstudium soll den Teilnehmenden vor allem digitales Wissen
näherbringen. Im ersten Semester lernen die Studierenden unter dem Motto
\glqq Technologische Skills\grqq{} die Grundlagen der Programmierung, sowie das
Verständnis von Internet of Things. Das zweite Semester befasst sich mit
\glqq Future Work Skills\grqq{} und bringt den Studierenden das Verständnis der
Themen \glqq Data Science\grqq{}, \glqq Digitale Ethik\grqq{},
\glqq Agile Working\grqq{}, sowie \glqq Coaching-Fähigkeiten\grqq{} bei. Im
letzten Semester haben die Lernenden die Möglichkeit ein eigenes
Digitalisierungsprojekt zu planen und umzusetzen.

\newpage

Die studienbegleitende Ausbildung wird den Studierenden nach Abschluss aller
Veranstaltungen in Form eines offiziellen Hochschulzertifikats angerechnet.

Um im ersten Semester die Grundlagen der Programmierung automatisiert und
trotzdem mit verständlichem Feedback zu lehren, wird eine zeit- und
ortsunabhängige, innovative Lernplattform benötigt. Die Teilnehmer sollen dabei
trotzdem eine möglichst realitätsnahe Programmierumgebung kennenlernen. Um solch
ein System nachhaltig bereitzustellen, wird viel Planung benötigt. Dazu gehört
eine Anforderungsanalyse, ein durchdachtes Konzept, sowie der Vergleich
mit bereits etablierten Plattformen. Genau mit dieser Aufgabe beschäftigt sich
der Kern dieser Arbeit.

\subsection{Struktur der Arbeit}\label{struktur-der-arbeit}
Die Arbeit besteht aus mehreren wesentlichen Bestandteilen.
\autoref{anforderungsanalyse} behandelt alle nötigen Anforderungen, die für den
Aufbau einer Online-Programmierplattform wichtig sind. Diese werden hierbei
unterteilt in funktionale und nichtfunktionale Anforderungen. Die funktionalen
Anforderungen werden dabei aus zwei unterschiedlichen Sichten dargestellt: die
Anforderungen aus der Sicht von Studierenden bzw. Teilnehmenden, sowie aus der
Sicht von Lehrenden.

\autoref{softwarearchitektur} wiederum vergleicht und diskutiert vorhandene
Lernsysteme. Hierbei werden neben den Vorteilen auch die Probleme beim Einsatz
an der Hochschule Regensburg besprochen. Zum Abschluss des Kapitels werden alle
aufgeführten Online-Lernplattformen anhand der vorher aufgestellten funktionalen
und nichtfunktionalen Anforderungen neutral mithilfe einer Nutzwertanalyse
bewertet und schließlich miteinander verglichen. Die finale Entscheidung,
welches System am besten zum Zusatzstudium Digital Skills passt, fällt auf das
System, welches bei der Nutzwertanalyse am meisten Punkte erreicht.

Die endgültige Entscheidung, sowie deren Konfiguration und Implementierung wird
hinterher in \autoref{konfiguration-u-impl} näher erläutert und beschrieben.
Dabei geht es vor allem um technische Details, wie z.B. zusätzliche Software,
die zur Unterstützung der Teilnehmenden entwickelt werden muss.

Anschließend wird die Programmierlernplattform in \autoref{studie} in Form
einer Feldstudie an nicht Informatik oder Mathematik studierenden Testprobanden
getestet. Am Anfang des Kapitels wird das Testkonzept und der Aufbau der Studie
erläutert. Danach folgt die Auswertung der Testergebnisse.

Schließlich fasst \autoref{zusammenfassung-u-ausblick} die Ergebnisse
zusammen und gibt einen weiteren Ausblick auf die Zukunft des Zusatzstudiums
Digital Skills, sowie auf den darin enthaltenen automatisierten Programmierkurs.