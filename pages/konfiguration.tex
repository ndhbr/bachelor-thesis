\section{Konfiguration und Implementierung}

% Tutors.dev
\subsection{Tutors als Aufgabensammlung}

% GitHub Classroom
\subsection{GitHub Classroom}
\subsubsection{Konfiguration}
Für die Einrichtung wurde eine GitHub Organisation erstellt. GitHub
Organisationen können von jedem Nutzer GitHub-Nutzer erstellt werden und
benötigen lediglich einen Namen und eine Kontakt-E-Mail-Adresse. Die
Organisation dieses Kurses beherbergt alle Aufgabenvorlagen, die später näher
erläuterte Vorlage für Replit, sowie alle Aufgabenrepositories der Studierenden.

\subsubsection{Erste Aufgaben}
Nach der Erstellung der OTH-Organisation mussten die Aufgabenvorlagen erstellt
werden. Aufgabenvorlagen sind in GitHub Classroom normale Repositories, welche
in GitHub als Vorlage markiert wurden. Sie beinhalten meist zusätzlich Tests, um
den Code darin zu prüfen.

Sobald die Organisation mit Aufgaben gefüllt war, konnte der
\glqq Classroom\grqq{} für den Kurs angelegt werden. Anfangs ist ein Classroom,
wie die Organisation auch, leer. Über die Oberfläche können neue Assignments
erstellt werden. Assignments sind Aufgaben, welche dem Studierenden zur
Verfügung stehen. Für jedes vorher angelegte Aufgabenrepository wurde ein
Assignment erstellt. Bei der Erstellung gibt man verschiedene
Konfigurationsparameter an. Dazu gehört die Auswahl, ob eine Aufgabe von
Einzelpersonen, oder einer Gruppe bearbeitet werden kann, oder ob die jeweiligen
Versuche für alle Studierenden oder nur für die Lehrer sichtbar sind. Ferner
gibt es die Möglichkeit eine Deadline, sowie den Starter Code anzugeben. Für den
Starter Code wurde in diesem Fall jeweils das Aufgabenrepository ausgewählt.
\subsubsection{Tests und Benotung}
Im nächsten Schritt legt man die Benotung und das Feedback fest. Das Autograding
(die Benotung) geschieht über Kommandos in der Konsole. In unserem Fall
beinhaltet jedes Aufgabenrepository einen Ordner mit pytest-Tests. Pytest ist
eine Code-Test-Bibliothek für Python. Die Assignments wurden so konfiguriert,
dass GitHub nach jedem Push zum Repository des Studierenden die pytest-Tests
gestartet werden. Wenn alle Tests erfolgreich sind, erhält der Studierende eine
vorkonfigurierte Punktzahl. Durch Classroom ist es außerdem möglich, jedem
Test eine individuelle Punktzahl zuzuweisen. So kann man dem Schüler neben der erfolgreichen Ausführung beispielsweise noch Bonuspunkte für das Fangen von
nicht geplanten Eingaben vergeben.

% Replit
\subsection{Erstellung eines Replit-Starter-Templates}
\subsubsection{Allgemein}
\subsubsection{OTH-Console}
\subsubsection{SSH-Keys}
\subsubsection{Wrapper-Tools (get, check, submit)}