\section{Einleitung und Motivation}\label{einleitung}
\subsection{Herausforderung digitales Lernen}\label{herausforderung}


\subsection{Kurs Digital Skills als Zusatzstudium}\label{kurs-digital-skills}
\subsection{Struktur der Arbeit}\label{struktur-der-arbeit}
Die Arbeit besteht aus mehreren wesentlichen Bestandteilen. Kapitel
\ref{anforderungsanalyse} behandelt alle nötigen Anforderungen, die für den
Aufbau einer digitalen Lernplattform wichtig sind. Kapitel
\ref{softwarearchitektur} wiederum vergleicht und diskutiert zuerst vorhandene
Lern-Systeme und trifft dann die finale Entscheidung für den Kurs Digital
Skills. Die finale Entscheidung und deren Konfiguration und Implementierung wird
dann folgend in Kapitel \ref{konfiguration-u-impl} näher erläutert und
beschrieben. Schließlich wird die Programmierlernplattform in Kapitel
\ref{studie} in Form einer kleinen Studie an nicht Informatik- oder Mathematik-
studierenden Testprobanden getestet. Schließlich fasst Kapitel
\ref{zusammenfassung-u-ausblick} alles zusammen und gibt einen weiteren Ausblick
auf die Zukunft des Systems.