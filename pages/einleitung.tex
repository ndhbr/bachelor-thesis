\section{Einleitung und Motivation}\label{einleitung}
\subsection{Herausforderung digitales Lernen}\label{herausforderung}
Deutschland gilt als sehr rückschrittlich im Thema Digitalisierung an Schulen
und Universitäten. Nicht zuletzt hat die Corona-Pandemie viele
Bildungseinrichtungen dazu gezwungen, umzudenken und digitale Lerninhalte
zu erstellen. Mit dieser Herausforderung kamen jedoch einige Einrichtungen und
Lehrende schnell an ihre Grenzen.

Professor Doktor Markus Heckner hat sich dieser Problemstellung gestellt und
hat gemeinsam mit Mitarbeitenden der Hochschule Regensburg an einer Lösung
gearbeitet. Das Ergebnis ist ein optionales Zusatzstudium, um die sogenannten
\glqq Digital Skills\grqq{} der Studierenden zu verbessern und zu intensivieren. 

\subsection{Kurs Digital Skills als Zusatzstudium}\label{kurs-digital-skills}
Digital Skills ist ein aus drei Semester bestehendes Zusatzstudium für alle
Studierenden, mit Ausnahme von Studierenden der Fakultät Informatik und
Mathematik, der Hochschule Regensburg.

Das Zusatzstudium soll den Teilnehmern vor allem digitales Wissen näher bringen.
In Semester 1 lernen die Studierenden unter dem Motto
\glqq Technologische Skills\grqq{} die Grundlagen der Programmierung, sowie das Verständnis von Internet of Things. Das zweite Semester befasst sich mit
\glqq Future Work Skills\grqq{} und bringt den Teilnehmern das Verständnis der
Themen Data Science, Digitale Ethike, Agile Working, sowie Coaching Fähigkeiten
bei. Im letzten Semester haben die Lernenden die Möglichkeit ein eigenes Digitalisierungsprojekt zu planen und umzusetzen.

Die studienbegleitende Ausbildung wird den Studierenden nach Abschluss aller
Veranstaltungen in Form eines Hochschulzertifikats, sowie anhand einer Erwähnung
im Abschlusszeugnis, angerechnet.

Um im ersten Semester die Grundlagen der Programmierung automatisiert und
trotzdem mit Feedback zu lehren, wird eine zeit- und ortsunabhängige innovative
Lernplattform benötigt. Genau mit dieser Aufgabe beschäftigt sich der Kern
dieser Arbeit.

\subsection{Struktur der Arbeit}\label{struktur-der-arbeit}
Die Arbeit besteht aus mehreren wesentlichen Bestandteilen. Kapitel
\ref{anforderungsanalyse} behandelt alle nötigen Anforderungen, die für den
Aufbau einer digitalen Lernplattform wichtig sind. Kapitel
\ref{softwarearchitektur} wiederum vergleicht und diskutiert zuerst vorhandene
Lern-Systeme und trifft dann die finale Entscheidung für den Kurs Digital
Skills. Die finale Entscheidung und deren Konfiguration und Implementierung wird
dann folgend in Kapitel \ref{konfiguration-u-impl} näher erläutert und
beschrieben. Schließlich wird die Programmierlernplattform in Kapitel
\ref{studie} in Form einer kleinen Studie an nicht Informatik- oder Mathematik-
studierenden Testprobanden getestet. Schließlich fasst Kapitel
\ref{zusammenfassung-u-ausblick} alles zusammen und gibt einen weiteren Ausblick
auf die Zukunft des Systems.