\section{Anforderungsanalyse}\label{anforderungsanalyse}
Funktionale Anforderungen beschreiben, welche Features und Funktionen das
Projekt bieten muss. Diese Informationen sind wichtig, um die richtigen
Werkzeuge und Programme für die Umsetzung auszuwählen. Dabei geht es meist um
sehr konkret formulierte Wünsche. Nichtfunktionale Bedingungen sind wiederum
Bedingungen, wie zum Beispiel Zuverlässigkeit, Verfügbarkeit oder diverse
Sicherheitsanforderungen. Sie lassen sich eher als Qualitätseigenschaften
beschreiben.

Die folgenden Anforderungen werden durch die wissenschaftliche Mitarbeiterin
Julia Ruhland und dem Studierenden Andreas Huber festgelegt. Die funktionalen
Anforderungen werden durch Andreas Huber aufgestellt und in wöchentlichen
Meetings mit Prof. Dr. Heckner verifiziert. Andreas Huber sammelt außerdem
gemeinsam mit Julia Ruhland User-Stories aus der Sicht der Studierenden und den
Dozierenden in der Online-Plattform Trello. Die nichtfunktionalen Anforderungen
werden wiederum von den User-Stories abgeleitet. % TODO: Anhang

\subsection{Funktionale Anforderungen}\label{anforderungsanalyse-funktional}
\subsubsection{Studierende}\label{anforderungsanalyse-funktional-stud}
Den Studierenden sollte eine Kursübersicht mit allen Aufgaben zur Verfügung
gestellt werden. Der persönliche Fortschritt der jeweiligen Aufgaben sollte
dabei leicht ersichtlich sein. Mögliche Deadlines oder Abgabefristen sollen bei
jeder Aufgabe deutlich erkennbar sein.

Des Weiteren muss es Studierenden möglich sein, ohne die Installation von
zusätzlichen Programmen, die Aufgaben online bearbeiten, prüfen und abgeben zu
können. Trotzdem sollen ihnen die Option offen stehen, die Aufgaben in der
Entwicklungsumgebung ihrer Wahl lösen zu können.

Eine weitere wichtige Anforderung bei der Prüfung der Aufgaben ist es, dass
die Bearbeiter der Aufgaben sogenannte \emph{human-readable} (für den Menschen
lesbare) Fehlermeldungen erhalten müssen. Das bedeutet, dass die Fehlermeldungen
bei der Überprüfung auch für fachfremde Studierende leicht verständlich sein
müssen. Fehlermeldungen bzw. konstruktives Feedback muss dabei automatisiert und
jederzeit generiert werden können.

Die Möglichkeit, Aufgabenversuche abzugeben, muss ebenfalls mit wenig Aufwand
behaftet sein. Falls ein Versuch fehlschlägt, oder nicht die volle Punktzahl
erreicht wird, sollte der Teilnehmende jederzeit die Möglichkeit haben, einen
neuen Versuch hochladen zu können.

\subsubsection{Lehrende}\label{anforderungsanalyse-funktional-lehrende}
Lehrende müssen neue Aufgaben anlegen und konfigurieren können, dazu gehört
unter anderem die Festlegung eines Abgabedatums.

Des Weiteren müssen Dozierende eine Übersicht an Aufgaben des jeweiligen Kurses
haben.

Einzelne Aufgaben sollten durch Lehrende temporär versteckt oder deaktiviert
werden können.

Darüber hinaus muss es möglich sein, mehrere Administratoren zu den Kursen
hinzuzufügen. Dadurch können verschiedene Personen Aufgaben erstellen und die
abgegebenen Lösungen herunterladen. Dies ist gleichzeitig die nächste
Anforderung: Administratoren müssen mit wenig Aufwand alle Abgaben der
Teilnehmenden herunterladen können.

Ferner müssen sowohl die Aufgaben, als auch die jeweiligen Deadlines nach der
Erstellung editierbar sein. Dadurch können Dozierende ihre Aufgaben stetig
verbessern.

Um einen Überblick über die Schwierigkeit der Aufgaben behalten zu können,
müssen Lehrende den Aufgaben-Fortschritt der Studierenden je nach Kurs
übersichtlich einsehen können. Sollte ein Großteil der Teilnehmenden an
einzelnen Aufgaben scheitern, kann die Aufgabenstellung im Einzelnen erneut
evaluiert und verbessert werden.

\subsection{Nichtfunktionale Anforderungen}
\label{anforderungsanalyse-nichtfunktional}
Das eingesetzte System muss neben den funktionalen Anforderungen auch diverse
nichtfunktionale Anforderungen erfüllen, um von der OTH als sinnvolle
Lernplattform eingesetzt werden zu können.

Der erste wichtige Punkt ist, dass die Lernplattform möglichst zuverlässig ist.
Zur Zuverlässigkeit gehört neben einer hohen Verfügbarkeit auch eine skalierende
Performance, wenn viele Studierende gleichzeitig die Plattform nutzen wollen.

Ein weiterer Punkt ist die Wartbarkeit. Die Plattform sollte mit möglichst wenig
Wartung sicher bestehen bleiben können. Außerdem sollte nur auf externe Systeme
gesetzt werden, von denen ausgegangen werden kann, dass diese noch einige Jahre
gepflegt werden. Hier empfiehlt sich ein Aufteilen der Plattform auf mehrere
externe Tools, um bei einem Ausfall oder Außerbetriebnahme eines einzelnen Tools
noch einen Notbetrieb gewährleisten zu können. Der Austausch gegen eine andere
neue Softwarekomponente gestaltet sich dadurch leichter.

Die Ressourcenlast und die damit verbundenen Kosten spielen eine weitere
wichtige Rolle bei der Entscheidungsfindung. Wenn Teile der Lernplattform auf
OTH-Servern gehostet werden müssen, sollten diese möglichst ressourcenschonend
sein. Serverressourcen sind teuer und können das Projekt im Zweifelsfall
unrentabel machen, wenn das System bei paralleler Nutzung durch mehrere
Studierende eine inadäquate Serverlast voraussetzen würde. Selbiges gilt für
Lizenzgebühren möglicher Tools und Werkzeuge.

Neben den Kosten ist es auch wichtig, dass die Plattform zusammen mit dem
\ac{lms} bzw. der Lernplattform der Hochschule arbeiten kann. Der Vorteil einer
LMS-Integration wird später im Kapitel \ref{code-freak} näher erläutert.

Zu guter Letzt ist es wünschenswert, dass der Programmierkurs bei der Auswahl
der Programmiersprache flexibel ist. So sollte es beispielsweise möglich sein,
dass die erste Aufgabe mit der Programmiersprache Java gelöst wird,
während die zweite mit Python gelöst werden muss.