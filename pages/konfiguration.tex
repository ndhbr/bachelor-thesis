\section{Konfiguration und Implementierung}\label{konfiguration-u-impl}

% Tutors.dev
\subsection{Tutors als Aufgabensammlung}\label{tutors-als-aufgabensammlung}
% TODO: Vielleicht nachfragen?
Da der Programmierkurs nur ein Teil eines ganzen Modulkatalogs des Kurses
Digital Skills ist, übernimmt die Konfiguration von Tutors die hierfür
zuständigen wissenschaftlichen Mitarbeiter des Zusatzstudiums.

Für die Einpflegung der Aufgaben werden lediglich Anleitungen sowie
jeweils dazugehörige Thumbnails (Vorschaubilder) benötigt. Die Anleitungen
werden im standardisierten Format Markdown verfasst. Das Format ist zum heutigen
Zeitpunkt in der Informatik sehr verbreitet.

% TODO: Markdown Quelle? Readme erklären?
Anleitungen können mit verhältnismäßig wenig Aufwand verfasst werden und
schließlich als \code{README.md}-Datei im Aufgaben-Repository abgelegt
werden. Dies hat den Vorteil, dass die Anleitung auch in der Versionskontrolle
der Aufgabe enthalten sind. Außerdem können GitHub und Replit beim Klick auf die
Aufgabe die Anleitung neben Tutors auch zusätzlich formatiert anzeigen.

% GitHub Classroom
\subsection{GitHub Classroom}\label{github-classroom}
\subsubsection{Konfiguration}\label{classroom-konfiguration}
Für die Einrichtung wurde eine GitHub Organisation erstellt. GitHub
Organisationen können von jedem Nutzer GitHub-Nutzer erstellt werden und
benötigen lediglich einen Namen und eine Kontakt-E-Mail-Adresse
\parencite{github-organisation-erstellen}. Die Organisation dieses Kurses beherbergt
alle Aufgabenvorlagen, die später näher erläuterte Vorlage für Replit, sowie
alle Aufgabenrepositories der Studierenden.

\subsubsection{Erste Aufgaben}\label{classroom-erste-aufgaben}
Nach der Erstellung der OTH-Organisation mussten die Aufgabenvorlagen erstellt
werden. Aufgabenvorlagen sind in GitHub Classroom normale Repositories, welche
in GitHub als Vorlage markiert wurden. Sie beinhalten meist zusätzlich Tests, um
den Code darin zu prüfen.

Sobald die Organisation mit Aufgaben gefüllt war, konnte der
\glqq Classroom\grqq{} für den Kurs angelegt werden. Anfangs ist ein Classroom,
wie die Organisation auch, leer. Über die Oberfläche können neue Assignments
erstellt werden. Assignments sind Aufgaben, welche dem Studierenden zur
Verfügung stehen. Für jedes vorher angelegte Aufgabenrepository wurde ein
Assignment erstellt. Bei der Erstellung gibt man verschiedene
Konfigurationsparameter an. Dazu gehört die Auswahl, ob eine Aufgabe von
Einzelpersonen, oder einer Gruppe bearbeitet werden kann, oder ob die jeweiligen
Versuche für alle Studierenden oder nur für die Lehrer sichtbar sind. Ferner
gibt es die Möglichkeit eine Deadline, sowie den Starter Code anzugeben. Für den
Starter Code wurde in diesem Fall jeweils das Aufgabenrepository ausgewählt.
\parencite{github-assignment-erstellen}
\subsubsection{Tests und Benotung}\label{classroom-tests}
Im nächsten Schritt legt man die Benotung und das Feedback fest. Das Autograding
(die Benotung) geschieht über Kommandos in der Konsole. In unserem Fall
beinhaltet jedes Aufgabenrepository einen Ordner mit pytest-Tests. Pytest ist
eine Code-Test-Bibliothek für Python. Die Assignments wurden so konfiguriert,
dass GitHub nach jedem Push zum Repository des Studierenden die pytest-Tests
gestartet werden. Wenn alle Tests erfolgreich sind, erhält der Studierende eine
vorkonfigurierte Punktzahl. Durch Classroom ist es außerdem möglich, jedem
Test eine individuelle Punktzahl zuzuweisen. So kann man dem Schüler neben der erfolgreichen Ausführung beispielsweise noch Bonuspunkte für das Fangen von
nicht geplanten Eingaben vergeben. \parencite{github-assignment-erstellen}

% Replit
\subsection{Erstellung eines Replit-Starter-Templates}\label{replit-template}
\subsubsection{Template-Repository}\label{replit-template-repository}
Projekte in Replit heißen Repls. Diese Repls können auf Basis von GitHub
Repositories erzeugt werden. Die Import-Funktionalität ermöglicht das
Bereitstellen von Dateien, die die Bearbeitung der Kursaufgaben erleichtern.

Aus diesem Grund wurde ein öffentliches Repository in der vorher erstellten
GitHub Organisation angelegt. Dies ist das Template, welches später von den
Studierenden als Starter-Vorlage verwendet wird. Replit versteckt Dateien,
welche sich in einem Ordner namens \texttt{/node\_modules} befinden.
Normalerweise wird der Ordner automatisch im Kontext mit externen Modulen der
JavaScript-Bibliothek Node.js verwendet \parencite{nodejs}. Das ist auch der Grund
weshalb Replit diesen Ordner automatisch versteckt. Dieses Verhalten nutzen wir,
um Hilfsprogramme und Konfigurationen zu verstecken.

Studierende werden den ganzen Kurs in einem Repl absolvieren. Jede
Programmieraufgabe wird als Ordner im Projekt-Repl abgespeichert werden. Um die
Aufgaben herunterladen, prüfen und schließlich abgeben zu können, benötigt
man git-Kenntnisse, sowie Erfahrungen mit Test-Frameworks, wie zum Beispiel 
pytest. Im \texttt{/node\_modules}-Ordner des Starter-Templates befinden sich
diverse Wrapper-Tools, welche dem Studierenden die Arbeit abnehmen und sie dabei
unterstützen.

\subsubsection{Wrapper-Tools (get, check, submit)}
\label{replit-template-wrapper-tools}
Die Wrapper-Tools \texttt{get}, \texttt{check} und \texttt{submit} sind Python
Skripte. Über den Konsolenbefehl \texttt{get <PROJEKT-REPOSITORY>} kann der
Studierende die Aufgabe in seinen Arbeitsbereich laden. Der Befehl 
\texttt{check <PROJEKT-REPOSITORY>} erlaubt es die Aufgabe auf Fehler zu
überprüfen.

Schließlich pusht der Befehl \texttt{submit <PROJEKT-REPOSITORY>}
den Lösungsversuch in das GitHub Classroom Aufgabenrepository des
Kursteilnehmers.

Damit die Skripte Ordnerunabhängig ausgeführt werden können, werden
Konsolenaliase benötigt. Konsolenaliase können in einer Bash-Konsole
beispielsweise über das Anhängen folgender Zeile an die \texttt{.bashrc}-Datei
erstellt werden: \texttt{alias befehl="echo Hallo"}. Bash ist eine Art
\glqq Standard-Shell\grqq{} unter Linux und wird auch von Replit als Konsole
eingesetzt \parencite{bash}. Bei jeder neuen Konsolensitzung wird dann der Alias aus
der Datei eingelesen und angewendet. Die genannte Datei befindet sich in der
Regel im Benutzerverzeichnis, welches außerhalb des Arbeitsbereiches in Replit
liegt. Alle Änderungen außerhalb des Arbeitsbereiches werden jedoch von Replit
nach jeder Sitzung zurückgesetzt. Um dieses Problem zu beheben, wurde die nun
folgende OTH-Console eingeführt.

\subsubsection{OTH-Console}\label{replit-template-oth-console}
Sobald der Student in seinem Repl auf den Run-Knopf drückt, startet die
sogenannte OTH-Console in der Konsole. Dies ist eine neue
modifizierte Konsoleninstanz, welche alle für die Arbeit benötigten
Konfigurationen enthält.

Sobald der Run-Knopf gedrückt wird, startet das Einrichtungsskript
\texttt{setup.py}, welches die benötigten Dateien in das Benutzerverzeichnis
schreibt.

Zuerst wird eine Konfigurationsdatei für Bash angelegt. In diese
werden alle benötigten Aliase (\texttt{get}, \texttt{submit}, \texttt{check}
und \texttt{github}) geschrieben. Außerdem wird GitHub, falls noch
nicht vorhanden, zu den sogenannten \glqq Known Hosts\grqq{} im SSH-Ordner
hinzugefügt. Der Vorteil daran ist, dass der Student bei der ersten Verbindung
mit GitHub (bspw. durch den \texttt{get}-Befehl) keine Authentizitätsprüfung
bestätigen muss \parencite{ssh}. Als letztes wird in der Konfigurationsdatei noch das
Aussehen des Bash Promptes festgelegt.

Im nächsten Schritt wird die passwortlose Authentifizierung mit GitHub
eingerichtet. Hierzu benötigt man ein SSH-Schlüssle-Paar, welches automatisch,
falls nicht vorhanden, durch das Einrichtungsskript erzeugt wird. Nach der
Erzeugung wird es neben dem SSH-Ordner auch in die Repl-Nutzer-Datenbank
geschrieben. Die Datebank ist ein simpler Key-Value-Speicher, welcher jeweils
pro Replit-Projekt existiert \parencite{replit-database}. Sobald der Studierende
Replit neustartet und das Benutzerverzeichnis gelöscht wurde, holt sich das Einrichtungsskript die SSH-Keys aus der Datenbank und schreibt sie wieder
zurück in die jeweiligen Dateien. Dasselbe Verfahren wird für die Konfiguration
von Git angewandt. Git benötigt, um Änderungen zu pushen, einen Namen mit
zugehöriger E-Mail Adresse \parencite{git-config}. Diese Daten werden zusammen mit
den SSH-Keys in der Replit-Datenbank gespeichert.

Nach der Ausführung des Einrichtungsskripts, wird eine neue
Bash-Konsolen-Instanz, mit der gerade angelegten Konfigurationsdatei als
Parameter, gestartet.

\subsubsection{SSH-Keys}\label{replit-template-ssh-keys}
Um das vorher generierte SSH-Schlüsselpaar für die Authentifizierung gegen
GitHub zu verwenden, muss der öffentliche Schlüssel noch zu dem GitHub Profil
des Studierenden hinzugefügt werden. Hierfür enthält das Starter-Template ein
weiteres Programm, welches mit dem Befehl \texttt{github} in der OTH-Console
ausgeführt werden kann. Dieses weitere Python Programm lädt den im SSH-Ordner
gespeicherten öffentlichen Schlüssel und gibt ihn zusammen mit einem Link
zum Hinzufügen von SSH-Keys in GitHub aus.

Desweiteren überprüft das Programm, ob Git bereits konfiguriert ist.
Sind die benötigten Werte nicht in der Replit-Datenbank vorhanden, frägt das
Programm den Nutzer nach dem Name und der studentischen E-Mail Adresse, welche schließlich beide in der Datenbank hinterlegt werden. Außerdem werden die Werte,
wie beim Einrichtungsskript auch, in eine Git-Konfigurationsdatei geschrieben.