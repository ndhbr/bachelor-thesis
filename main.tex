% Festlegung des allgemeinen Dokumentenformats
\documentclass[a4paper,12pt,headsepline]{article}

% Schrift
\usepackage[T1]{fontenc}
\usepackage{lmodern}
\usepackage[utf8]{inputenc}
\usepackage[ngerman]{babel}

% Bilder
\usepackage{graphicx}

% Variablen
% Variablen
\newcommand{\titleDocument}{Bachelorarbeit}
\newcommand{\subjectDocument}{im Studiengang Informatik}
\newcommand*{\bildquelle}{%
  \footnotesize Quelle:
}

% mehrseitige Tabellen ermöglichen
\usepackage{longtable}

% Packet für Seitenrandabstände und Einstellung für Seitenränder
\usepackage{geometry}
\geometry{left=3.5cm, right=2.5cm, top=2.5cm, bottom=2cm}

% bricht lange URLs "schön" um
\usepackage[hyphens,obeyspaces,spaces]{url}

% Festlegung Art der Zitierung - Havardmethode: Abkuerzung Autor + Jahr
\bibliographystyle{alphadin}

% Paket für Zeilenabstand
\usepackage{setspace}
\onehalfspacing

% für Bildbezeichner
\usepackage{capt-of}

% für Stichwortverzeichnis
\usepackage{makeidx}

% Für Phantomsection
\usepackage{hyperref}

% Konfiguriere das Inhaltsverzeichnis
\usepackage{tocbasic}
\DeclareTOCStyleEntries[
  raggedentrytext,
  numwidth=0pt,
  numsep=1ex,
  dynnumwidth,
]{tocline}{chapter,section,subsection,subsubsection,paragraph,subparagraph}
\DeclareTOCStyleEntries[
  indent=0pt,
  linefill=\TOCLineLeaderFill,
]{tocline}{section,subsection,subsubsection,paragraph,subparagraph}

% Titel
\title{Bachelorarbeit}

% Autor
\author{Andreas Huber}

% Datum
\date{\today}

%
% Start
% des
% Dokuments
%
\begin{document}

% Titelseite
\thispagestyle{empty}

\begin{figure}[t]
 \centering
 \includegraphics[width=0.4\textwidth]{assets/oth/logo}
\end{figure}

\begin{verbatim}
\end{verbatim}

\begin{center}
    \Large{Ostbayerische Technische Hochschule Regensburg}
\end{center}

\begin{center}
    \Large{Fakultät für Informatik}
\end{center}

\begin{verbatim}
\end{verbatim}

\begin{center}
    \doublespacing
    \textbf{\LARGE{\titleDocument}}\\

    \singlespacing

    \begin{verbatim}
    \end{verbatim}

    \textbf{{~\subjectDocument}}
\end{center}

\begin{verbatim}
\end{verbatim}

\begin{verbatim}
\end{verbatim}

\begin{center}
    \textbf{zur Erlangung des akademischen Grades \\ Bachelor of Science}
\end{center}

\begin{verbatim}
\end{verbatim}

\begin{flushleft}
    \begin{tabularx}{\linewidth}{@{}>{\bfseries}l@{\hspace{.9em}}X@{}}
        \textbf{Thema:}         & Analyse, Konzeption und Implementierung eines Tools für das automatisierte Testen, Einreichen und Benoten von Programmieraufgaben \\
                                & \\
        \textbf{Vorgelegt von:} & Andreas Huber <andreas.huber@st.oth-regensburg.de> \\
        \textbf{Matrikelnummer:}& 3180161 \\                                                                                                                                                                               \\
                                & \\
        \textbf{Version vom:}   & \today \\
        \textbf{Erstgutachter:} & Prof. Dr. Markus Heckner \\
        \textbf{Zweitgutachter:}& Prof. Dr. Johannes Schildgen \\
    \end{tabularx}
\end{flushleft}
\newpage

% Römische Nummerierung
\pagenumbering{roman}

% Inhaltsverzeichnis
\tableofcontents
\newpage

% Arabische Seitennummerierung ab hier
\pagenumbering{arabic}

% Einleitung
\section{Einleitung und Motivation}\label{einleitung}
\newpage

% Römische Nummerierung
\pagenumbering{roman}
\setcounter{page}{3}

% Literaturliste soll im Inhaltsverzeichnis auftauchen
% \phantomsection
\addcontentsline{toc}{section}{Literaturverzeichnis}
% Literaturverzeichnis anzeigen
\renewcommand\refname{Literaturverzeichnis}
\printbibliography
% \bibliography{assets/literatur}
\newpage

\end{document}