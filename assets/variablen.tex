% Variablen
\newcommand{\titleDocument}{Bachelorarbeit}
\newcommand{\subjectDocument}{Analyse, Konzeption und Implementierung eines Tools für das automatisierte Testen, Einreichen und Benoten von Programmieraufgaben}
\newcommand*{\bildquelle}{
  \footnotesize Quelle:
}
\newcommand{\code}[1]{\noindent\ignorespaces\texttt{#1}}

% Code
\usepackage{listings}
\usepackage{color}

\definecolor{black}{rgb}{0,0,0}
\definecolor{dkgreen}{rgb}{0,0.6,0}
\definecolor{gray}{rgb}{0.5,0.5,0.5}
\definecolor{mauve}{rgb}{0.58,0,0.82}

\lstset{literate=%
    {Ö}{{\"O}}1
    {Ä}{{\"A}}1
    {Ü}{{\"U}}1
    {ß}{{\ss}}1
    {ü}{{\"u}}1
    {ä}{{\"a}}1
    {ö}{{\"o}}1
    {~}{{\textasciitilde}}1
}

\lstdefinestyle{Bash} {
  frame=single,
  language=Bash,
  aboveskip=3mm,
  belowskip=3mm,
  showstringspaces=false,
  columns=flexible,
  basicstyle={\small\ttfamily},
  numbers=none,
  numberstyle=\tiny\color{black},
  keywordstyle=\color{black},
  commentstyle=\color{black},
  stringstyle=\color{black},
  breaklines=true,
  breakatwhitespace=true,
  tabsize=3
}

\lstset{emph={\$},emphstyle=\textbf}

\lstdefinestyle{Python} {
  frame=single,
  language=Bash,
  aboveskip=3mm,
  belowskip=3mm,
  showstringspaces=false,
  columns=flexible,
  basicstyle={\small\ttfamily},
  numbers=left,
  numberstyle=\tiny\color{mauve},
  keywordstyle=\color{mauve},
  commentstyle=\color{dkgreen},
  stringstyle=\color{dkgreen},
  breaklines=true,
  breakatwhitespace=true,
  tabsize=2
}