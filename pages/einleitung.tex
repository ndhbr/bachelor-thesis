\section{Einleitung und Motivation}\label{einleitung}
\subsection{Herausforderung digitales Lernen}\label{herausforderung}
Deutschland gilt als sehr rückschrittlich im Thema Digitalisierung an Schulen
und Universitäten. Nicht zuletzt hat die Corona-Pandemie den Rückstand
Deutschlands in vielen Bereichen offengelegt. Neben Verwaltungen, Unternehmen,
Schulen oder Gerichten hat die Pandemie vor allem viele Hochschulen dazu
gezwungen, umzudenken und digitale Lerninhalte zu erstellen. Mit dieser
Herausforderung kamen jedoch einige Einrichtungen und Lehrende ohne jegliche
Vorbereitungen schnell an ihre Grenzen. \parencite{bmwi-rueckstand}

Professor Doktor Markus Heckner hat sich dieser Problemstellung gestellt und
hat gemeinsam mit Mitarbeitenden der Hochschule Regensburg an einer Lösung
gearbeitet. Das Ergebnis ist ein optionales Zusatzstudium, um die sogenannten
\glqq Digital Skills\grqq{} der Studierenden zu verbessern und zu intensivieren.
Langfristig gesehen sollen die Zusatzmodule die digitalen Fähigkeiten der
Teilnehmer fördern und somit zur Einholung des digitalen Rückstands in
Deutschland beitragen.

\subsection{Kurs Digital Skills als Zusatzstudium}\label{kurs-digital-skills}
Digital Skills ist ein aus drei Semester bestehendes Zusatzstudium für alle
Studierenden, mit Ausnahme von Studierenden der Fakultät Informatik und
Mathematik, der Hochschule Regensburg. Wie der Name bereits sagt, findet das
Zusatzstudium parallel zu dem Hauptstudium der Studierenden statt.

Das Zusatzstudium soll den Teilnehmern vor allem digitales Wissen näher bringen.
In Semester 1 lernen die Studierenden unter dem Motto
\glqq Technologische Skills\grqq{}
die Grundlagen der Programmierung, sowie das Verständnis von Internet of Things.
Das zweite Semester befasst sich mit \glqq Future Work Skills\grqq{} und bringt
den Teilnehmern das Verständnis der Themen Data Science, Digitale Ethike, Agile
Working, sowie Coaching Fähigkeiten bei. Im letzten Semester haben die Lernenden
die Möglichkeit ein eigenes Digitalisierungsprojekt zu planen und umzusetzen.

% TODO: ist das so?
Die studienbegleitende Ausbildung wird den Studierenden nach Abschluss aller
Veranstaltungen in Form eines Hochschulzertifikats, sowie anhand einer Erwähnung
im Abschlusszeugnis, angerechnet.

Um im ersten Semester die Grundlagen der Programmierung automatisiert und
trotzdem mit verständlichem Feedback zu lehren, wird eine zeit- und
ortsunabhängige innovative Lernplattform benötigt. Die Teilnehmer sollen dabei
trotzdem eine möglichst realitätsnahe Programmierumgebung kennenlernen. Um solch
ein System nachhaltig bereitzustellen, benötigt man viel Planung und Verständnis
der Studierenden. Genau mit dieser Aufgabe beschäftigt sich der Kern dieser
Arbeit.

\subsection{Struktur der Arbeit}\label{struktur-der-arbeit}
Die Arbeit besteht aus mehreren wesentlichen Bestandteilen. Kapitel
\ref{anforderungsanalyse} behandelt alle nötigen Anforderungen, die für den
Aufbau einer online Programmierplattform wichtig sind. Die Anforderungen werden
hierbei unterteilt in funktionale und nichtfunktionale Anforderungen. Die
funktionalen Anforderungen werden dabei aus zwei unterschiedlichen Sichten
dargestellt: die Anforderungen aus der Sicht von Studierenden bzw. Teilnehmern,
sowie aus der Sicht von Lehrenden.

Kapitel \ref{softwarearchitektur} wiederum vergleicht und diskutiert zuerst
vorhandene Lern-Systeme. Hierbei werden neben den Vorteilen auch die Probleme
beim Einsatz an der Hochschule Regensburg besprochen. Zum Abschluss des Kapitels
werden alle aufgeführten Online-Lernplattformen anhand der vorher aufgestellten
funktionalen und nichtfunktionalen Anforderungen neutral mithilfe einer Tabelle
bewertet und schließlich miteinander verglichen. Die finale Entscheidung,
welches System am besten zu dem Zusatzstudium Digital Skills passt, fällt auf
das System, welches bei dem Tabellenvergleich am meisten Punkte erreicht hat.

Die finale Entscheidung, sowie deren Konfiguration und Implementierung wird
dann folgend in Kapitel \ref{konfiguration-u-impl} näher erläutert und
beschrieben. Dabei geht es vor allem um technische Details, wie z.B. zusätzliche
Software, die zur Unterstützung der Teilnehmer entwickelt werden muss.

Anschließend wird die Programmierlernplattform in Kapitel \ref{studie} in Form
einer Feldstudie an nicht Informatik- oder Mathematik- studierenden
Testprobanden getestet. Am Anfang des Kapitels wird das Testkonzept und der
Aufbau der Studie erläutert. Danach folgt die Auswertung der Testergebnisse.

Schließlich fasst Kapitel \ref{zusammenfassung-u-ausblick} alles zusammen und
gibt einen weiteren Ausblick auf die Zukunft des Zusatzstudiums Digital Skills,
sowie auf den darin enthaltenen automatisierten Programmierkurs.