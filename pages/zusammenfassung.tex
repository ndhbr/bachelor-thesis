\section{Zusammenfassung und Ausblick}\label{zusammenfassung-u-ausblick}
Nach der abgeschlossenen Studie folgt jetzt eine Zusammenfassung. Die
Aufgabenstellung war die Analyse, Konzeption und Implementierung eines Tools
für das automatisierte Testen, Einreichen und Benoten von Programmieraufgaben.
Diese Aufgabenstellung konnte nach erfolgreicher Analyse und Vergleich
bestehender Online-Programmierplattformen mithilfe von GitHub-Classroom als
Werkzeug erfolgreich absolviert werden.

Neben GitHub Classroom wird Replit als Online-Entwicklungsumgebung eingesetzt.
Um die Studierenden zu unterstützen, bekommen sie Zugang zu einem
GitHub-Repository, welches als Vorlage für die Einrichtung eines Workspaces
dient. In diesem Replit Workspace können die Studierenden alle
Programmieraufgaben des Zusatzstudiums erledigen. Damit die Bearbeitung leichter
fällt, enthält das Template außerdem verschiedene Skripte. Darunter enthalten
sind Aufgabenmanagement-Tools, welche es erlauben, die Aufgaben herunterzuladen,
zu überprüfen und abzugeben, sowie ein Einrichtungsskript für GitHub und eine
eigene angepasste Konsoleninstanz. Die Aufgabenstellungen werden zusammen mit
den anderen Aufgaben des Kurses auf der Plattform Tutors gehostet.

Das Zusatzstudium Digital Skills kann durch den Einsatz dieser
Lernplattform sehr profitieren. Sowohl bei den Korrekturen, als auch bei den 
Aufgabenstellungen werden die zuständigen Dozierenden entlastet. Sie müssen
lediglich bereit sein, wenn Studierende bei der Bearbeitung der Übungen Fragen
haben, zu helfen. Die Materialien aller Fächer des Zusatzstudiums befinden sich
auf der genannten Plattform Tutors. Durch die Einbindung der
Programmierplattform in die vorhandenen Materialien wurde ein einheitliches
Lernsystem geschaffen. Doch auch die Teilnehmenden werden von der
Programmierplattform profitieren. Alle Module und Aufgaben von Digital Skills
befinden sich sortiert an einem gesammelten Ort. Dadurch müssen sich Studierende
nicht in diversen Plattformen registrieren und können leichter die Übersicht
behalten. Das selbstständige Lösen der Aufgaben unterstützt außerdem den
Lernprozess und intensiviert die Lerninhalte.

Die durchgeführte Feldstudie war ein Erfolg. Schwierige Stellen und Hürden
konnten gefunden werden und können nun für den produktiven Einsatz der Plattform
angepasst werden. Die Ergebnisse des Interviews zeigten, dass durchaus einige
Teilnehmenden Interesse an dem Zusatzstudium haben. Außerdem deckte die Studie
auf, dass selbst einfache Python-Aufgaben ohne exakt durchdachte Anleitungen
für fachfremde Personen schnell Probleme bereiten können.

Den technologischen Rückstand Deutschlands kann das Zusatzstudium sicherlich
nicht alleine aufholen. Trotzdem ist der Kurs Digital Skills zusammen mit der
dafür entwickelten Lösung, um online Programmieraufgaben lösen zu können, ein
guter Schritt in die richtige Richtung. Das Zusatzstudium hilft, zukünftige
Generationen besser auf die digitale Welt vorzubereiten.